\documentclass{article}
\usepackage{amsmath}
\usepackage{enumitem}
\title{HW \#2}
\author{
    Uziel Rivera-Lopez
}
\date{\today}
\begin{document}
\maketitle

\section*{Question 1}
\begin{enumerate}[label=\alph*)]
    \item  Event A is drawing a red ball, $\frac{8}{20}$, event B is drawing a ball either red or green so $\frac{8}{20} + \frac{5}{20} = \frac{13}{20}$, and event C
    is not drawing a blue ball so $1-\frac{7}{20}=\frac{13}{20}$.

    P(B $\cap$ C) = $P(B)*P(C)$ = $\frac{13}{20} \cdot \frac{13}{20} = \frac{169}{400}$
    \\P(A$|$B $\cap$ C) = $\frac{P(A \cap B \cap C)}{P(B \cap C)} = \frac{P(A)}{P(B \cap C)} = \frac{\frac{8}{20}}{\frac{169}{400}} = \frac{1600}{3380} = \frac{800}{1690}$

    \item Because if you were doing B$\cap$C given A, and A given B$\cap$C
    P(A$|$B$\cap$C) $\neq$ P(B$\cap$C $|$ A) 
    they do not equal each other, hence they're dependent events from each other. Also if you were to do something like P(A$\cap$B$\cap$C) and P(A)* P(B$\cap$C) you would get different results.

    \item $P(A \cap B) = P(B|A)P(A)$
    \\$P(A | B) = \frac{P(A \cap B)}{P(B)} = \frac{P(B|A) P(A)}{P(B)}$
    \\ By using the formula, we can see that the probability of A given B is dependent on the probability of B given A.
\end{enumerate}

\section*{Question 2}
\begin{enumerate}[label=\alph*)]
    \item Pr(p=0.5) = 0.9
        \\Pr(p=0.6) = 0.1
        \\$Pr(5H) = Pr(5H|p=0.5) \cdot Pr(p=0.5) + Pr(5H| p=0.6) * Pr(p=0.6)= $
        \\ $(\binom{5}{5}*0.5^5) \cdot 0.9 + (\binom{5}{5}*0.6^5) \cdot 0.1$ = 0.035901
        \\Pr(p=0.5$|$5H) = $\frac{Pr(5H) \cdot Pr(p=0.5)}{Pr(5H)} = \frac{0.28125}{ 0.035901} = .7834 = Pr(p=0.5)$
        \\Pr(p=0.6) = 1 - Pr(p=0.5) = 0.2166
    \item $\beta(a,b) = x^{a-1} \cdot (1-x)^{b-1} =  p^{a-1} \cdot (1-p)^{b-1}$ So,
    \\$\beta(2,2) = x \cdot (1-x) =  p \cdot (1-p)$
    \\$ $Likelihood$ = (1-p)^{10}$
    \\$ $Posterior$ = p^{2-1}*(1-p)^{2-1} \cdot (1-p)^{10} = p^{2-1}*(1-p)^{12-1} = \beta(2,12)$ 

\end{enumerate}

\section*{Question 3}
\begin{enumerate}
    \item $L(\Theta)=\prod_{i=1}^{n} \frac{1}{\sqrt[]{2\pi\Theta_1}}\cdot e^{-\frac{(x_i-\Theta_0)^2}{2\Theta_1}}$ 
    \\$log(L(\Theta)) = -\frac{n}{2}log\Theta_1 -\frac{n}{2}log(2\pi)-\frac{\Sigma(x_i - \Theta_0)}{2\Theta_1}$
    \\$L(\Theta_0) = -\frac{n}{2}log\Theta_1 -\frac{n}{2}log(2\pi)-\frac{\Sigma(x_i - \Theta_0)}{2\Theta_1} $ 
    \begin{enumerate}[label=\Roman*]
        \item $\frac{d}{d\Theta_0} L(\Theta_0) = \frac{\Sigma (x_i - \Theta_0)}{\Theta_1} \cdot \Theta_1$
        \item $\Theta_0 = \hat{\mu} = \frac{\Sigma x_i}{n} = \frac{\Sigma X}{n} = EX$
    \end{enumerate}
    $L(\Theta_0) = -\frac{n}{2}log\Theta_1 -\frac{n}{2}log(2\pi)-\frac{\Sigma(x_i - \Theta_0)}{2\Theta_1} $ 
    \begin{enumerate}[label=\Roman*]
        \item $\frac{d}{d\Theta_1} L(\Theta_1) = -\frac{n}{2\Theta_1} + \frac{\Sigma(x_i - \Theta_0)^2}{2\Theta_1^2}$
        \item $\frac{n}{2\Theta_1} = \frac{\Sigma(x_i - \Theta_0)^2}{2\Theta_1^2}$
        \item $\Theta_1= \hat{\sigma}^2 = \frac{\Sigma(x_i - \Theta_0)^2}{n} =  \frac{\Sigma(X - EX)^2}{n}$
    \end{enumerate}
    \item $\Theta_0 = \hat{\mu} = \frac{\sum_{i=1}^{10} X}{10} = \frac{-5 + 7 + 9 + 12 - 17 + 1 + 4 + 24 + 0 + 3}{10} = 2.8=  EX$
    \\ $\Theta_1= \hat{\sigma}^2 = \frac{\sum_{i=1}^{10}(X - 2.8)^2}{10} = \\\frac{(-5 - 2.8)^2 + (7 - 2.8)^2 + (9 - 2.8)^2 + (12 - 2.8)^2 + (-17 - 2.8)^2 + (1 - 2.8)^2 + (4 - 2.8)^2 + (24 - 2.8)^2 + (0 - 2.8)^2 + (3 - 2.8)^2}{10}\\= 28.36$
\end{enumerate}

\section*{Question 4}
\begin{enumerate}
    \item $L(p|x) = \prod_{i=1}^{n} p^{x_i} \cdot (1-p)^{1-x_i}$
    \\\\To maximize the likelihood funcion we take the derivative of the natural log likelihood function, which kind of also means logging and 
    deriving P(x). since we are deriving the equation for we p we want to set it as p.
    \begin{enumerate}[label=\Roman*]
        \item  $L(p) = p^{\Sigma x_i} \cdot (1-p)^{n-\Sigma (x_i)}$
        \item  $log(L(p)) = \Sigma x_i \cdot log(p) + \Sigma (1-x_i) \cdot log(1-p)$
        \item $\frac{d}{dp}(L(p)) = \frac{\Sigma x_i}{p} - \frac{\Sigma x_i - n}{1-p}$
        \item $(\frac{\Sigma x_i}{p} - \frac{\Sigma x_i - n}{1-p}) \cdot p(1-p)$
        \item $\Sigma x_i-(\Sigma x_i \cdot p) - n+ (\Sigma x_i \cdot p)$
        \item $\hat{p} = \frac{\sum_{i=1}^{n} X}{n}$
    \end{enumerate}
    \item Since we have two dice that have 1-6 on them, then the only times it equals 9 is 6+3, 3+6, 5+4, and 4+5, so basically four possiblites.
    \\$\hat{p} = \frac{\sum_{i=1}^{36} X}{36}$
    \\Since we have 36 possible outcomes, and 4 of them are 9.
    \\$\hat{p} = \frac{4}{36} = \frac{1}{9}$
\end{enumerate}

\section*{Question 5}
$P(X = 5) =  P(X=5 | \Theta=2) \cdot P(\Theta=2) + P(X=5|\Theta=4) \cdot P(\Theta=4) = 0.0469$
\\$P(\Theta =2 \cap X = 5) = 0.022 \cdot 0.7 = 0.0154$
\\$P(\Theta =4 \cap X = 5) = 0.105 \cdot 0.3 = 0.0315$
\\$P(\Theta = 2 | X=5) = \frac{P(\Theta =2 \cap X = 5)}{X=5} =0.3284$
\\$P(\Theta = 4 | X=5) = \frac{P(\Theta =4 \cap X = 5)}{X=5} =0.6716$


\section*{Question 6}
\begin{enumerate}
   \item With the Bayes’ Estimator, we have the data and have some knowledge of the prior information/distribution. Having both of these two things, we can use this to calculate the posterior distribution. 
   \item By knowing the prior information/distribution, it influences the posterior distribution. This is because the prior information is used to calculate the posterior distribution, as well as update the degree of certainty. 
\end{enumerate}


\end{document}